\chapter{関連研究}

本章では,本研究と同様に,仮想化技術を用いてマルチディスプレイシステムを構築した関連研究をいくつか紹介し,その特徴について説明する.

大規模ディスプレイシステムにおいて動画再生時のフレームレート向上を試みた例としては,Bundulisらの研究 \cite{bundulis2018infiniviz}がある.Bundulisらは,以前にInfiniviz \cite{bundulis2016infiniviz}とよばれる大規模ディスプレイシステムを提案している.Infinivizは,仮想マシンベースの高解像度ディスプレイウォールシステムであり,既存の他の大規模ディスプレイしシステムと比較して,ネットワーク帯域幅の消費量と計算性能を向上させようとするものである.また,シームレスな方法で可視化タスクにアプローチし,大規模な高解像度ディスプレイウォール上で,一般的なデスクトップオペレーティングシステムのソフトウェアを変更することなく実行できるようにすることを目的としている.Bundulisらの研究では,Quake 3 Arena \cite{quake3arena}を解像度9600x5400, 24fpsで動作させた際のInfinivizの実際の性能を計測した.この研究では,仮想化を利用することで,ソフトウェアに依存しない仮想マシンベースの高解像度ディスプレイウォールシステムを構築している.

\begin{figure}[H]
    \hspace*{\fill}
    \includegraphics[width=90mm]{./fig/chap5/Infiniviz.eps}
    \hspace*{\fill}
    \caption{Infinivizのアーキテクチャ}
\end{figure}

\clearpage

Liuらは,DockerとGPUを利用して、画像、動画、WebGLアプリケーションなどを表示する大画面ビジュアライゼーションアプリケーションをデプロイする手法を提案している \cite{Liu_2019}.
この研究では,Dockerクラスタを構築して28個のコンテナを含むマイクロサービスを起動し,Unix/Linuxユーザにグラフィカルなインタフェースを提供するために広く使用されているX11Unixソケットをマッピングすることで大画面可視化システムをデプロイした.
同時に,アプリケーションをローカルホストで動作させた場合と比較し,性能を検証し,彼らの手法を大画面ビジュアライゼーションに適用した場合,許容できる性能劣化で良好な結果を得ることができると結論付けている.

\begin{figure}[H]
    \hspace*{\fill}
    \includegraphics[width=\linewidth]{./fig/chap5/docker_cluster.eps}
    \hspace*{\fill}
    \caption{Dockerクラスタを用いて構築したマルチディスプレイのトポロジ}
\end{figure}