\chapter{小型低性能計算機を用いたマルチディスプレイシステムの構築}

本章では,小型低性能計算機,特にSBCを利用した低コストなマルチディスプレイ構築手法の必要性と,その実現に向けて行われた先行研究について説明する.
まず2.1節では,従来利用されてきたマルチディスプレイの構築手法とそのコストについて述べる.
続く2.2節で先行研究でも使用されており,本研究でも取り扱うシングルボードコンピュータ (SBC) について説明し,その後2.3節で先行研究である,SBCを用いて構築されたマルチディスプレイシステムについて説明する.
さらに2.4節で,先行研究で提案されたSBCを用いたマルチディスプレイシステムについての問題点と,それを改善するための技術的な課題点について述べる.

\section{従来のマルチディスプレイ構築手法}
現在マルチディスプレイの構築には数多くの手法が提案されており,それらは大きく分けると2種類の手法に分類することができる\cite{pccluster,6693038}.
1つはマルチディスプレイを構築可能にする専用のハードウェアを用いて構築する手法,もう1つはマルチディスプレイを構築可能にする専用のミドルウェアを用いて構築する手法である.
本節では,この2種類の構築手法を簡単に解説してそれぞれの構成例を示し,構築に要する金銭的コストについて説明する.

\subsection{専用ハードウェアを用いたマルチディスプレイの構築}

マルチディスプレイ構築用のハードウェアは接続した複数台のディスプレイに対して映像出力用の信号を送信する機能を持った機器であり,主にデジタルサイネージ\cite{signage}や会議用のディスプレイを構築する際に利用される.
マルチディスプレイ構築用ハードウェアの多くは,HDMI (High-Definition Multimedia Interface), DisplayPort \cite{displayport}等の規格に対応した映像出力ポートを複数搭載しており,各ポートに映像信号を分配することによって,接続したディスプレイを連動させて制御する.
マルチディスプレイ構築用ハードウェアは,表示映像フレームの分割,接続ディスプレイ間の同期処理などを高速に行うことが可能であるため,高フレームレートでの映像出力が可能である.
一方で,マルチディスプレイ構築用ハードウェアは入出力ポート数の上限が決まっているため,接続ディスプレイ数が制限される.
そのため,マルチディスプレイシステムを構築するディスプレイ数にも上限が存在することになり,スケーラビリティの観点からすると欠点となっている.
マルチディスプレイ構築用ハードウェアの具体例としては,グラフィックボードやマトリックススイッチャが存在する.
図2.1に,マトリックススイッチャを用いて構築したマルチディスプレイの概要図を示す.

\begin{figure}[htbp]
 \includegraphics[width=1.02\linewidth]{./fig/matrixswitcher.eps}
 \caption{システムの動作フロー}
 \label{fig.1}
\end{figure}

グラフィックボードは,PCなどのコンピュータにおいて,映像を信号として入出力するための機能を拡張ボードとして独立させたものである.
各グラフィックボードに搭載されているチップやメモリによって,描画可能な解像度や表現色数,2D/3D描画性能などが異なる.
マルチディスプレイを構築する目的でグラフィックボードを使用すると,ある程度の高解像な画像を描画可能であり,信号の入出力可能ポート数が多層備え付けられているものが必要となる.
マルチディスプレイ対応型のグラフィックボードは,NVIDIA社やAMD社から発売されている.
マルチディスプレイ対応型のグラフィックボードは,一般的に利用可能な映像出力ポート数が多く,表示解像度が優れているものほど価格が高くなる.

\subsection{専用ミドルウェアを用いたマルチディスプレイの構築}

マルチディスプレイ構築用ミドルウェアは,ディスプレイに接続した汎用PCに画像フレームまたは描画命令を送信することによって動画や画像を表示する.
マルチディスプレイ構築用ミドルウェアは主に科学的可視化の分野で広く利用されており,高性能なコンピュータで行われたシミュレーションの結果を表示する大規模可視化装置の構築などに使用される.
接続可能なディスプレイ数に上限は設けられておらず,多数の汎用PCを連携して動作させることで非常にスケーラブルなマルチディスプレイの構築が可能である.
しかし,マルチディスプレイ構築用ミドルウェアは大規模な科学的可視化の用途を想定して設計されたものが多く,ディスプレイと接続された汎用PC群に対して高負荷な描画処理を要求する.
そのため,マルチディスプレイを構築するために使用される汎用PCはある程度高い性能を持ったものでなければならず,マルチディスプレイを構成するディスプレイ数が増加するごとに構築費用が高騰するという問題点がある.
図2.2に,マルチディスプレイ構築用ミドルウェアを用いて汎用PC群を連携させることで構築したマルチディスプレイの構成例を示す.

\begin{figure}[htbp]
 \includegraphics[width=1.02\linewidth]{./fig/middleware.eps}
 \caption{システムの動作フロー}
 \label{fig.1}
\end{figure}

それぞれのPCはマルチディスプレイ構築用ミドルウェアによってヘッドノードまたはディスプレイノードとして動作することが可能になる.ヘッドノードは各ディスプレイに画像を表示するための信号の送信やヘッドノード-ディスプレイノード間の同期処理などの役割を担う.ディスプレイノードはマルチディスプレイを構築するディスプレイに接続され画像の描画や表示に関する計算や処理を行う.
ディスプレイノードとディスプレイは1対1で接続されるため,ディスプレイノードはマルチディスプレイを構築するディスプレイの数と同数にする必要がある.

以下,専用ミドルウェアを用いたマルチディスプレイシステムの詳細な動作について説明する.
各ミドルウェアによって行われる連携表示処理には,大きく分けて2通りの手法がある.
1つはフレーム転送方式と呼ばれる手法である.
フレーム転送方式はヘッドノードがディスプレイノードに対して表示映像のフレームを送信し,ヘッドノードとディスプレイノードの同期処理によってフレームをディスプレイ上に連携表示させる方式である.
フレーム転送方式を採用するマルチディスプレイの例として,SAGE \cite{sage}やSAGE2 \cite{sage2}などがあげられる.

フレーム転送方式は,以下の5種類の動作から成り立つ.

\begin{itemize}
    \item フレーム圧縮
    \item フレーム送信
    \item フレーム展開
    \item 同期制御
    \item フレーム表示
  \end{itemize}

まずヘッドノード上で表示映像からフレームが切り出され,フレームの圧縮処理が行われる.この時,フレームを各ディスプレイノードの表示領域に合わせて分割する手法と,フレーム全体をそのまま圧縮する手法の2通りが存在する.
前者はSAGE \cite{sage},後者はDisplayCluster \cite{displaycluster}といったミドルウェアでそれぞれ採用されている.
フレームの圧縮にはDXT (DirectX Texture Compression) \cite{dxt}やJPEG (Joint Photographic Experts Group) \cite{jpeg}などが利用される.

続いて,ヘッドノードで圧縮されたフレームを各ディスプレイノードへ送信する.
ディスプレイノードは,ヘッドノードから圧縮フレームを受信したあと展開処理を行い,続いて同期処理を開始する.
同期処理では,まずフレームの展開を終え表示待機状態になったディスプレイノードからヘッドノードへ表示準備完了通知が送信される.
ヘッドノードは,接続された全てのディスプレイノードから表示準備完了通知を受け取った後,各ディスプレイノード宛に表示命令を送信する.
表示命令を受け取ったディスプレイノードは,該当フレームをディスプレイ上に表示する.
この一連の動作を繰り返し行うことで,ディスプレイ上に連携して動画を表示することが可能になる.

もう1つは,描画命令転送方式と呼ばれる方式である.この方式では,ヘッドノードはディスプレイノードに対して画像フレームの送信を行わず,アプリケーションの出力する描画命令を同期処理を行いつつ送信することで動画の連携表示を行う.この方式を採用しているミドルウェアとしては,DMX (Distributed Multihead X) やChromium \cite{chromium}などがあげられる.

描画命令転送方式は,主に以下の4種類の動作から成り立つ.

\begin{itemize}
    \item 描画命令のキャプチャリング
    \item 描画命令転送
    \item 同期制御
    \item フレーム描画
\end{itemize}

  まず,ヘッドノードはアプリケーションから出力された描画命令のキャプチャリングを行う.
  続いて,ヘッドノードがキャプチャリングした描画命令をディスプレイノードへ送信する.
  描画命令を受信したディスプレイノードは,命令を受け取ったことを通知するメッセージをヘッドノードへ送信する.
  ヘッドノードは全てのディスプレイノードから描画命令受信通知を受信し次第,フレーム描画命令を送信し,ディスプレイノードがこの描画命令を受信しディスプレイ上にフレームを描画する.
  この一連の動作を繰り返し行うことで,ディスプレイ上に連携して動画を表示する.

\section{シングルボードコンピュータ(SBC)}

シングルボードコンピュータ(SBC)は,1枚の基板上に必要最低限の部品を取り付けることで構築された小型計算機である.
SBCは他の汎用PCと比較しても非常に低価格で手に入れることができ,消費電力も少ないため,IoT (Internet of Things) 向けの機器として一般に利用される\cite{130007722836,7380571}.
また,軽量プログラミング言語が利用できることから,教育用やプログラミング学習用としても使用されている.
SBCにはRaspberry Piシリーズを代表として,様々な種類のものが存在する.
表2.1に主なSBCの機種とその仕様を示す.
また,図2.3に主なSBCの画像を示す.

SBCが持つCPUの特徴としては,主に低周波数のコアがデュアルコア,クアッドコアなどの形式で複数搭載されているものが多い.
これは消費電力を抑えるため,また価格を低く抑えるためであり,こうした省電力や低価格という特徴を維持しつつCPU自体の処理能力を向上させる目的で,近年ではコア数の増加が進んでおり,マルチコア方式を採用した機種が多く販売されるようになってきている.
搭載メモリや通信帯域に関しては,近年のSBCの性能向上により4.0 GB以上の比較的大きなメモリ容量を持つものも増えており,ギガビットイーサネットに対応したNICを有するものも多く発売されるようになってきている.
映像出力の機能としては,ほとんどの機種でHDMI端子による出力が可能であり,Full HD 解像度での出力が標準的である.
また,各機種のOSには軽量なLinuxベースのOSが広く採用されている.
例えばRaspberry Piに対応したOSであるRaspbian \cite{raspbian}もLinuxをベースとするOSの一種である.



\section{先行研究で提案されたマルチディスプレイシステム}

本節では,先に述べた先行研究で提案された,SBCを用いて構築されたマルチディスプレイの特徴について述べる.
まず,マルチディスプレイを構築する各ノードの動作について説明する.
その後フレームの連携表示処理を可能にする仕組みについて概説する.

\subsection{連携表示処理}

先行研究で構築されたマルチディスプレイシステムは,1台のヘッドノードと,SBCを用いて実装された複数台のディスプレイノードで構成されている.
システムの概要を図2.4に示す.

このマルチディスプレイシステムは,2.1.2節で述べた2種類の方式のうち,フレーム転送方式を採用している.
そのため,SBCをディスプレイノードとして用いることとなり,従来のフレーム転送方式をそのまま適用するとSBC上で負荷の高い処理を行う必要がある.
しかし,SBCは市販の汎用PCなどと比べると処理速度,描画性能などの面で劣っているため,可能な限りディスプレイノード上での処理負荷を小さく抑えなければならないという制約が存在する.
そのため,SBCがディスプレイノードとして十分高速に動作するような設計を目的として,JPEG圧縮や連携表示処理のパイプライン化などが実装されている.

以下では,先行研究において提案されたマルチディスプレイシステムの詳細な動作について述べる.
ヘッドノードでは,動画ファイルからフレームを切り出した後,接続ディスプレイノード数に応じてフレームの分割を行う.
分割したフレームはそれぞれJPEG方式で圧縮され,各ディスプレイノードへと送信される.
ヘッドノードから送信された圧縮フレームを受信したディスプレイノードでは,フレームを展開し,その展開が終了した後表示バッファに格納してヘッドノードに表示準備完了メッセージを送信する.ヘッドノードは全てのディスプレイノードからの表示準備完了メッセージを受け取り次第表示命令を送信し,命令を受け取ったディスプレイノードはディスプレイ上にフレームを表示する.

また,ノード上の各処理はスレッド処理によって行われており,ヘッドノード上では圧縮スレッド,送信スレッド,同期制御スレッドの3種類,ディスプレイノード上では受信スレッド,展開スレッド,表示制御スレッドの3種類のスレッドが動作している.
各スレッドの操作を図2.5に示す.
圧縮スレッドは,表示映像の切り出しと分割,圧縮までを行う.
送信スレッドは,圧縮されたフレームを各ディスプレイノードに送信する.
同期制御スレッドは,表示準備完了メッセージの受信と表示命令の送信を行う.
受信スレッドは,ヘッドノードから圧縮フレームを受信する.
展開スレッドは,圧縮フレームの展開を行う.
表示スレッドは,ヘッドノードからの表示命令を受け取りフレームをディスプレイ上に表示する.
   

\subsection{フィードバック処理}

JPEG圧縮処理では,まず入力画像のピクセルごとの画素値をRGB形式からYCbCr形式に変換する.
YCbCrは色彩表現法の一種であり,色を人間の目が認識しやすい輝度成分と認識しづらい色差成分とに分けて表現する\cite{YCbCr}.
YCbCr形式に変換された色成分はデータ量を削減するために間引かれることになるが,この時YCbCrサンプル比という間引き方を決定するパラメータを変更することで,
JPEG圧縮に要する時間を変更することができる~\cite{jpeg2}.
また,量子化を行う際の品質係数と呼ばれるパラメータを変更することによっても,JPEG圧縮に要する時間を短縮することができる.
先行研究のマルチディスプレイシステムでは,ディスプレイノードで動画再生時のフレームレートを計算し,目標フレームレートとの差に応じてこれらのパラメータを変更することで,フレームレートを目標値に近づけるフィードバック処理を実装している.





\section{問題点と技術的課題}

本節では,前節で述べた先行研究におけるMDの問題点と,それを解決するための技術的課題について述べる.
まず,先行研究のMDにおける具体的な問題について説明する.その後,問題点解決に向けた手法の考察と利点,欠点について述べ,比較を行う.

\section{先行研究で提案されたマルチディスプレイシステムの問題点}