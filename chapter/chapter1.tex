\chapter{序論}

近年では,デジタル化の進展により大量のデータを取得できるようになり,それらのデータを活用して研究機関が行う学術研究が盛んになっている.
学術研究においては,個々の研究機関の枠組みを越えて研究設備や研究データを利活用することでより大きな成果が期待できる.
このような背景から,大学や研究機関の研究者たちによる共同研究のプロジェクト数が増加傾向にある~\cite{nisted}.

また,最近では新たなスーパーコンピュータの開発による計算能力の向上を背景に,シミュレーションベースで行われる研究・プロジェクトの数が増加している.KBDD (K supercomputer-Based Drug Discovery project by Biogrid pharma consortium) プロジェクト\cite{kbdd,kbddproject}は,新薬の開発を目指して立ち上げられた共同研究プロジェクトであり,タンパク質と化合物の結合可能性の予測に高性能計算機による大規模シミュレーションの結果が活用されている.国立研究開発法人海洋研究開発機構 (JAMSTEC) は,東京大学地震研究所と協力して「富岳」を用いた大規模数値シミュレーションによる地震・津波による複合災害の統合的予測システムを開発している.また,同様に「富岳」を用いて行われたCOVID-19の飛沫・エアロゾル拡散モデルシミュレーションに関する共同研究も記憶に新しい.

以上に挙げたような大規模シミュレーションを用いた共同研究では,行われたシミュレーションの結果を可視化し,複数人の研究者たちで議論を交わすことが必要となる.
そのため,複数人で同時に閲覧できるような大型の高解像ディスプレイが不可欠である.
高解像のディスプレイを構築する方法として,格子状に配置した複数のディスプレイを1台の仮想ディスプレイとして構成するマルチディスプレイシステム(MD)がある.
MDは4Kなどの高解像な動画や画像を大画面で表示することが可能であるため,多数の研究者らが議論する場などで利用される.
MDの導入事例を挙げると,大阪大学サイバーメディアセンターでは24面ディスプレイで構成されるMDを吹田本館に,15面のMDをうめきたVisLabに設置し運用しており,大規模シミュレーションやデータ同化シミュレーションの解析結果を可視化する高解像度ディスプレイとして活用している\cite{ciber_media}.
また名古屋大学情報基盤センターでは,16面MDを用いた可視化装置と連動する大規模情報システムを運用している\cite{nagoya}.

しかし,MDの構築および運用コストは高くなる傾向にある.
MDの構築方法には,主に2通りの手法が存在する.
まずマトリックススイッチャを用いたMD構築用ハードウェアを用いる方法であるが,可視化用途の専用機器は一般に高価であるため,MD構築にかかる費用が高くなる.
もう一方はSAGE2~\cite{sage2}などに代表されるMD構築用ミドルウェアを用いる方法である.
この手法では,ディスプレイに映像を出力するための計算機を複数台用意し,それらとは別の制御用の計算機を用意する構成となる.
ディスプレイを接続した計算機では,3Dレンダリングや描画を行うため,一般的に高価なグラフィックボードを用いる.
また,制御用計算機では画像もしくは可視化データの送受信が頻繁に発生するため,高性能なネットワーク機器が用いられる.
そのため,可視化用途の計算機の費用は高くなる.

こうしたMDの構築費用に関する問題に対して,先行研究でシングルボードコンピュータ(SBC)を用いて構築したMDが提案された.
先行研究で提案されたMDは,画像フレームの分割・圧縮・送信,通信の制御を行うヘッドノードと,分割された画像フレームを描画するディスプレイノードとで構成されている.
まず,ディスプレイノードにSBCを用いるため,ディスプレイノードに高負荷な処理が要求されないようフレーム転送方式による連携表示処理を行っている.
ヘッドノードはフレームの分割および圧縮を行ってからフレームを送信し,ディスプレイノードが圧縮済フレームを受信し展開することで動画をディスプレイ上に表示する.

しかし,このMDではヘッドノード内でのフレーム処理を単一のプロセスで行っているため,高解像度な動画や高フレームレートな動画を表示する場合には動作に遅延が生じる.
また,MDの構成ディスプレイ数を増加させた場合にも動作遅延が生じるため,スケーラビリティにも問題を抱えている.
そこで本研究では画像フレームの圧縮を行うプロセスを構成ディスプレイと同じ数だけ用意し,ヘッドノード内で並列化してフレーム圧縮処理を行う手法を提案する.
フレーム処理の並列化はOS仮想化基盤を利用し,プロセスをコンテナとして動作させることで実現する.
フレーム処理の並列化を行うことで,ヘッドノード内でのフレーム処理時間を短縮とMDのスケーラビリティ向上を目指す.

以下,本論文の構成について述べる.
2章では,主なマルチディスプレイシステムの構築方法について説明し,低価格なマルチディスプレイ構築方法の重要性について述べる.また,ディスプレイノードにSBCを用いて構築された先行研究を紹介し,その関連研究として自身の卒業研究にも触れる.
3章では,本研究の目的を実現するための提案手法である,OS仮想化技術を用いたヘッドノードのフレーム処理並列化について説明し,その設計と実装,関連技術について詳細に説明する.
4章では,提案手法の効果を確認するための評価実験について述べる.
5章では,仮想化技術を用いてマルチディスプレイシステムを構築した関連研究について述べる.
最後に6章で本研究を総括し,今後の課題点について述べる.